\documentclass[letterpaper]{article}
\usepackage{natbib,alifexi}
\usepackage[T1]{fontenc}
\usepackage{lmodern}
\usepackage[utf8]{inputenc}
\usepackage{abstract}
\usepackage{textcomp}

\title{Étude d\textquotesingle algorithmes pour la détection de la tonalité de fichiers musicaux et implémentation en Clojure}
\author{Antoine Passemiers$$ \\
\mbox{}\\
$$Université Libre de Bruxelles \\
apassemi@ulb.ac.be}

\begin{document}
\maketitle

\renewcommand{\abstractname}{Résumé}    % Change the abstract title
\renewcommand\bibname{Bibliographie}        % Change the bib title
\renewcommand{\refname}{Bibliographie}

\begin{abstract}

Le projet consiste en la discussion de différents algorithmes relatifs à la détection 
automatisée de tonalité de fichiers musicaux, l\textquotesingle implémentation de ceux-ci en Python,
ainsi qu\textquotesingle une réflexion sur la comptabilité de ces derniers avec le paradigme fonctionnel.
Le choix de l\textquotesingle algorithme a concevoir selon l\textquotesingle approche fonctionnelle sera basé
sur des critères de rapidité d\textquotesingle exécution et de précision de la détection. 
L\textquotesingle algorithme final sera alors implémenté en Clojure.

\end{abstract}

\section{Introduction}

La tonalité d\textquotesingle une oeuvre musicale se caractérise par
l\textquotesingle ensemble des sons formant une même gamme diatonique. 
A la différence de la gamme, où les sons se succèdent de façon contigüe,
la tonalité (ou ton) regroupe des sons qui peuvent être disjoints et/ou superposés \citep{AD}.
En conséquence, nous nous intéressons à l\textquotesingle analyse de mélodies polyphoniques, 
où plusieurs notes peuvent être jouées en même temps.

En particulier, nous allons nous pencher sur deux catégories d\textquotesingle algorithmes :
ceux basés sur des modèles cognitifs, et ceux incluant des notions d\textquotesingle 
apprentissage automatique. Les premiers tentent d\textquotesingle intégrer au mieux les connaissances
de la théorie musicale et reposent sur la façon dont les personnes reconnaissent les différentes tonalités,
alors que les autres utilisent l\textquotesingle inférence statistique pour déterminer celles-ci.

L\textquotesingle approche cognitive utilisée pour la détection de la tonalité repose en partie
sur la solution proposée par Ibrahim Sha\textquotesingle ath lors 
de la conception du logiciel KeyFinder \citep{IS}. La précision de la détection est évaluée 
à l\textquotesingle aide d\textquotesingle une base 
de données, constituée de 250 fichiers musicaux au format wav, dont les tonalités sont connues
et inscrites dans un fichier csv. Ces fichiers font partie de ceux utilisés par 
Ibrahim Sha\textquotesingle ath dans le cadre de sa recherche.

Pour ce qui est de la partie apprentissage automatique, les méthodes présentées seront principalement
en lien avec les modèles de Markov cachés. Leur évaluation se fera en divisant la base de données en
un jeu de données d\textquotesingle apprentissage et un jeu de données de validation (respectivement
60 \% et 40 \% du jeu de données d\textquotesingle origine). Ce mini-mémoire se voulant concis et 
centré sur les objectifs décrits, le lecteur est supposé déjà disposer de connaissances suffisantes en 
apprentissage automatique, en théorie musicale et en traitement logiciel de signaux.

TODO : \citep{SP} \citep{AT}

\section{Considérations théoriques}

\subsubsection{Modèle cognitif}

Le signal audio est premièrement extrait du fichier wav, puis la moyenne entre les 
deux canaux est effectuée si le fichier a été enregistré en stéréo. En effet il n\textquotesingle
est pas nécessaire de prendre en compte le panoramique puisque celui-ci n\textquotesingle a que 
peu d\textquotesingle influence sur la mélodie dans le domaine spectral. 
Étant donné que les notes jouées sont uniquement caractérisées par leur fréquence fondamentale,
il n\textquotesingle est pas nécessaire de considérer l\textquotesingle entièreté du spectre
du fichier musical. De fait, la fréquence d\textquotesingle échantillonnage est abaissée à un dixième 
de la fréquence standard (4410 Hz), mais ce sous-échantillonnage est susceptible de provoquer des
phénomènes d\textquotesingle aliasing. Contrairement à la solution de Sh\textquotesingle ath, 
qui gère les problèmes d\textquotesingle aliasing sonore par l\textquotesingle application 
d\textquotesingle un filtre passe-bas, une approche plus simpliste et plus rapide se limiterait à
l\textquotesingle application d\textquotesingle une moyenne mobile sur une fenêtre glissante
de taille arbitraire. L\textquotesingle avantage de la méthode est de bénéficier d\textquotesingle effets
passe-bas sans devoir effectuer de calculs dans le domaine fréquentiel. Pour ce qui est de la taille
de la fenêtre temporelle, il s\textquotesingle agit d\textquotesingle un paramètre fixé durant
l\textquotesingle évaluation de l\textquotesingle algorithme.

TODO - suite

\subsubsection{Méthodes basées sur l\textquotesingle auto-corrélation}

Les fonctions d\textquotesingle auto-corrélation sot cool car indépendantes des phases du spectre.

\subsubsection{Modèles de Markov Cachés (HMM)}

Les modèles de Markov cachés sont des machines à états discrets cherchant à représenter des séries multivariées par
leurs distributions, ainsi que par les probabilités de transition entre les états cachés de la machine, où chaque état
caché possède ses propres probabilités d\textquotesingle émission. Contrairement à des modèles d\textquotesingle
apprentissage automatique plus populaires tels que les réseaux de neurones ou les machines à vecteurs de support,
les HMM sont capables de traiter des séquences de longueur variable.

TODO : \citep{JP} \citep{DR}

\subsubsection{Modèles de Markov Cachés de type entrée-sortie (IO-HMM)}

TODO : \citep{YB}

\subsubsection{Résultats}

\begin{table}[h]
\center{
\begin{tabular}{|c|c|c|c|c|c|}\hline
Méthode & ACC & REL & PAR & OBF & TOT \\ \hline\hline
COGN & 0,307  & 0,148 & 0,095 & 0,042 & 0,593 \\
EAA & -- & -- & -- & -- & -- \\
HMM & -- & -- & -- & -- & -- \\
IO-HMM & -- & -- & -- & -- & -- \\ \hline
\end{tabular}
}
\vskip 0.25cm
\caption{Évaluation des méthodes présentées selon différents indices :
ACC (accuracy), REL (relative keys), PAR (parallel keys) et OBF (out-by-a-fifth keys).
Le tableau reprend les scores du modèle cognitif (COGN), du modèle d\textquotesingle autocorrélation,
des modèles de Markov cachés (HMM), et du modèle de Markov caché d\textquotesingle entrée-sortie (IO-HMM).}
\end{table}

\section{Implémentation en Clojure}

TODO : \citep{SK}


\footnotesize
\bibliographystyle{apalike}
\bibliography{thesis}


\end{document}