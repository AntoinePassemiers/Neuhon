\documentclass[letterpaper]{article}
\usepackage{natbib,alifexi}
\usepackage[T1]{fontenc}
\usepackage{lmodern}
\usepackage[utf8]{inputenc}
\usepackage{abstract}
\usepackage{textcomp}
\usepackage{amsmath} 

\title{Étude d\textquotesingle algorithmes pour la détection de la tonalité de fichiers musicaux et implémentation en Clojure}
\author{Antoine Passemiers$$ \\
\mbox{}\\
$$Université Libre de Bruxelles \\
apassemi@ulb.ac.be}

\begin{document}
\maketitle

\renewcommand{\abstractname}{Résumé}    % Change the abstract title
\renewcommand\bibname{Bibliographie}        % Change the bib title
\renewcommand{\refname}{Bibliographie}



\begin{abstract}

Le projet consiste en la discussion de différents algorithmes relatifs à la détection 
automatisée de tonalité de fichiers musicaux, l\textquotesingle implémentation de ceux-ci en Python,
ainsi qu\textquotesingle une réflexion sur la comptabilité de ces derniers avec le paradigme fonctionnel.
Le choix de l\textquotesingle algorithme a concevoir selon l\textquotesingle approche fonctionnelle sera basé
sur des critères de rapidité d\textquotesingle exécution et de précision de la détection. 
L\textquotesingle algorithme final sera alors implémenté en Clojure.

\end{abstract}

\section*{1. Introduction}

La tonalité d\textquotesingle une oeuvre musicale se caractérise par
l\textquotesingle ensemble des sons formant une même gamme diatonique. 
A la différence de la gamme, où les sons se succèdent de façon contigüe,
la tonalité (ou ton) regroupe des sons qui peuvent être disjoints et/ou superposés \citep{AD}.
En conséquence, nous nous intéressons à l\textquotesingle analyse de mélodies polyphoniques, 
où plusieurs notes peuvent être jouées en même temps.\\

En particulier, nous allons nous pencher sur deux catégories d\textquotesingle algorithmes :
ceux basés sur des modèles cognitifs, et ceux incluant des notions d\textquotesingle 
apprentissage automatique. Les premiers tentent d\textquotesingle intégrer au mieux les connaissances
de la théorie musicale et reposent sur la façon dont les personnes reconnaissent les différentes tonalités,
alors que les autres utilisent l\textquotesingle inférence statistique pour déterminer celles-ci.\\

L\textquotesingle approche cognitive utilisée pour la détection de la tonalité repose en partie
sur la solution proposée par Ibrahim Sha\textquotesingle ath lors 
de la conception du logiciel KeyFinder \citep{IS}. La précision de la détection est évaluée 
à l\textquotesingle aide d\textquotesingle une base 
de données, constituée de 250 fichiers musicaux au format wav, dont les tonalités sont connues
et inscrites dans un fichier csv. Ces fichiers font partie de ceux utilisés par 
Ibrahim Sha\textquotesingle ath dans le cadre de sa recherche.\\

Pour ce qui est de la partie apprentissage automatique, les méthodes présentées seront principalement
en lien avec les modèles de Markov cachés. Leur évaluation se fera en divisant la base de données en
un jeu de données d\textquotesingle apprentissage et un jeu de données de validation (respectivement
60 \% et 40 \% du jeu de données d\textquotesingle origine). Ce mini-mémoire se voulant concis et 
centré sur les objectifs décrits, le lecteur est supposé déjà disposer de connaissances suffisantes en 
apprentissage automatique, en théorie musicale et en traitement logiciel de signaux.

TODO : \citep{SP} \citep{AT} -> Partie 2.1.1.

\section*{2. Considérations théoriques}

\subsection*{2.1. Pré-traitement et analyse spectrale}

\subsubsection*{2.1.1. Constant-Q Transform (CQT)}

Le signal audio est premièrement extrait du fichier wav, puis la moyenne entre les 
deux canaux est effectuée si le fichier a été enregistré en stéréo. En effet il n\textquotesingle
est pas nécessaire de prendre en compte le panoramique puisque celui-ci n\textquotesingle a que 
peu d\textquotesingle influence sur la mélodie dans le domaine spectral. 
Étant donné que les notes jouées sont uniquement caractérisées par leur fréquence fondamentale,
il n\textquotesingle est pas nécessaire de considérer l\textquotesingle entièreté du spectre
du fichier musical. De fait, la fréquence d\textquotesingle échantillonnage est abaissée à un dixième 
de la fréquence standard (4410 Hz), mais ce sous-échantillonnage est susceptible de provoquer des
phénomènes d\textquotesingle aliasing. Contrairement à la solution de Sh\textquotesingle ath, 
qui gère les problèmes d\textquotesingle aliasing sonore par l\textquotesingle application 
d\textquotesingle un filtre passe-bas, une approche plus simpliste et plus rapide se limiterait à
l\textquotesingle application d\textquotesingle une moyenne mobile sur une fenêtre glissante
de taille arbitraire. L\textquotesingle avantage de la méthode est de bénéficier d\textquotesingle effets
passe-bas sans devoir effectuer de calculs dans le domaine fréquentiel. Pour ce qui est de la taille
de la fenêtre temporelle, il s\textquotesingle agit d\textquotesingle un hyper-paramètre fixé durant
l\textquotesingle évaluation de l\textquotesingle algorithme. \citep{IS}\\

Ensuite, une fenêtre de Blackman est appliquée sur les données observées afin d\textquotesingle éviter les distortions spectrales
dues à la taille limitée de la fenêtre (et au principe d\textquotesingle incertitude d\textquotesingle Heisenberg). Le spectre est alors
approximé à l\textquotesingle aide de la transformée de Fourier rapide (FFT). L\textquotesingle intuition derrière la Constant-Q Transform (CQT)
est de penser que les coefficients de la FFT dont les fréquences ne correspondent pas à des notes de musique prennent plus de poids que les autres
coefficients. Ceci doit être réajusté en appliquant des fenêtres spectrales centrées sur les notes de musiques. Pour chaque fenêtre, les coefficients résultants de cette opération sont alors sommés pour ne former qu\textquotesingle un seul coefficient spectral. L\textquotesingle ensemble des coefficients globaux
constitue alors la CQT.


\subsubsection*{2.1.2. Méthodes basées sur l\textquotesingle auto-corrélation}

Les algorithmes reposant sur l\textquotesingle utilisation de fonctions d\textquotesingle auto-corrélation sont plus souples
car l\textquotesingle auto-corrélation peut être calculé de diverses manières, et seulement pour les fréquences voulues.
En effet, choisir le lag permet de fixer la fréquence voulue et, par extension, il est possible d\textquotesingle analyser
l\textquotesingle auto-corrélation pour toutes les notes musicales situées entre 65 Hz et 1109 Hz en évaluant la valeur
de la fonction d\textquotesingle auto-corrélation pour tous les lags correspondants.Le lag est alors simplement égal à la période
voulue. Un autre avantage des fonctions d\textquotesingle auto-corrélation est que celles-ci sont indépendantes des phases du spectre.

\subsubsection*{2.1.3. Estimation spectrale par moindres carrés}

Il est possible d\textquotesingle estimer le spectre par régression, au travers d\textquotesingle algorithmes n\textquotesingle exigeant
pas de grandes quantités de calcul. Parmi les algorithmes les plus importants résident celui de Vaníček et celui de Lomb-Scargle. Leur particularité est
de pouvoir traiter des données qui ne sont pas reçues à intervalles réguliers (la majorité des observations se font la nuit). En outre, contrairement à la transformée de Fourier rapide qui possède une résolution aussi précise que la fenêtre d\textquotesingle observation est grande, ces algorithmes
calculent un périodogramme dont la taille est fixée par le nombre de fréquences qui nous intéressent. Au plus le nombre de fréquences analysées est
élevé, au plus le temps d\textquotesingle exécution sera conséquent.

\begin{align}
\hat{\theta}_{k} = (A_{k}^{T} A_{k})^{-1} A_{k}^{T} y
\end{align}

Selon la méthode de Vanícek, le signal est supposé être centré sur zéro et représenter une combinaison linéaire de sinusoïdes et d\textquotesingle un bruit blanc.
Les coefficients de cette combinaison linéaire sont réunis dans un vecteur $\theta_{k}$, de telle manière que le signal équivaut à $y \approx A\theta_{k}$.
La matrice A est telle que chaque ligne de celle-ci constitue une sinusoïde de fréquence d\textquotesingle intérêt. Les coefficients sont alors finalement
donnés par l\textquotesingle équation 1. \citep{PV}\\

Le défaut de la méthode est de ne pas considérer les phases des fréquences concernées. En effet chaque sinusoïde contenue dans la matrice A
est supposée posséder une phase nulle. La méthode de Lomb-Scargle permet de résoudre ce problème en pré-calculant les phases des sinusoïdes
utilisées et en tenant compte au mieux de celles-ci lors du calcul du spectre. Le déphasage $\tau$ correspondant à la fréquence $t$ est donné par
l\textquotesingle équation 2. \citep{LS}

\begin{align}
\tan 2\pi f \tau = \frac{\sum\limits_{j=1} \sin 2\pi f t_{j}}{\sum\limits_{j=1} \cos 2\pi f t_{j}}
\end{align}

\begin{align}
\Delta R(f) = \frac{(YC)^{2}}{CC} 
+ \frac{(YS)^{2}}{SS}
\end{align}

\begin{align}
YC = \sum\limits_{j=1} y_{j}\cos 2\pi f (t_{j} - \tau)
\end{align}

\begin{align}
YS = \sum\limits_{j=1} y_{j}\sin 2\pi f (t_{j} - \tau)
\end{align}

\begin{align}
CC = \sum\limits_{j=1} \cos^{2} 2\pi f (t_{j} - \tau)
\end{align}

\begin{align}
SS = \sum\limits_{j=1} \sin^{2} 2\pi f (t_{j} - \tau)
\end{align}

\subsubsection*{2.1.4. Autres méthodes}

TODO : Cepstre, spectre de puissance, ...

\subsection*{2.2. Prédiction de la tonalité}

\subsubsection*{2.2.1. Modèle cognitif}

\subsubsection*{2.2.2. Modèles de Markov Cachés (HMM)}

Les modèles de Markov cachés sont des machines à états discrets cherchant à représenter des séries multivariées par
leurs distributions, ainsi que par les probabilités de transition entre les états cachés de la machine. De plus, chaque état
caché de cette dernière possède ses propres probabilités d\textquotesingle émission. Contrairement à des modèles d\textquotesingle
apprentissage automatique plus populaires tels que les réseaux de neurones ou les machines à vecteurs de support,
les HMM sont capables de traiter des séquences de longueur non fixée. Cette caractéristique est appréciable dans le cadre
de l\textquotesingle analyse de morceaux de musique, qui ont des durées de nature très variables.

TODO : \citep{JP} \citep{DR}

\subsubsection*{2.2.3. Modèles de Markov Cachés de type entrée-sortie (IO-HMM)}

TODO : \citep{YB}

\subsubsection*{2.2.4. Résultats}

\begin{table}[h]
\center{
\begin{tabular}{|c|c|c|c|c|c|}\hline
Méthode & ACC & REL & PAR & OBF & TOT \\ \hline\hline
COGN & 0,307  & 0,148 & 0,095 & 0,042 & 0,593 \\
EAA & -- & -- & -- & -- & -- \\
HMM & -- & -- & -- & -- & -- \\
IO-HMM & -- & -- & -- & -- & -- \\ \hline
\end{tabular}
}
\vskip 0.25cm
\caption{Évaluation des méthodes présentées selon différents indices :
ACC (accuracy), REL (relative keys), PAR (parallel keys) et OBF (out-by-a-fifth keys).
Le tableau reprend les scores du modèle cognitif (COGN), du modèle d\textquotesingle autocorrélation,
des modèles de Markov cachés (HMM), et du modèle de Markov caché d\textquotesingle entrée-sortie (IO-HMM).}
\end{table}

\section{Implémentation en Clojure}

TODO : \citep{SK}


\footnotesize
\bibliographystyle{apalike}
\bibliography{thesis}


\end{document}