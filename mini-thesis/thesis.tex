\documentclass[letterpaper]{article}
\usepackage{natbib,alifexi}
\usepackage[T1]{fontenc}
\usepackage{lmodern}
\usepackage[utf8]{inputenc}
\usepackage{abstract}
\usepackage{textcomp}

\title{Étude d\textquotesingle algorithmes pour la détection de la tonalité de fichiers musicaux et implémentation en Clojure}
\author{Antoine Passemiers$$ \\
\mbox{}\\
$$Université Libre de Bruxelles \\
apassemi@ulb.ac.be}

\begin{document}
\maketitle

\renewcommand{\abstractname}{Résumé}    % Change the abstract title
\renewcommand\bibname{Bibliographie}        % Change the bib title
\renewcommand{\refname}{Bibliographie}

\begin{abstract}

Le projet consiste en la discussion de différents algorithmes relatifs à la détection 
automatisée de tonalité de fichiers musicaux, la conception d\textquotesingle 
un de ces algorithmes à l\textquotesingle aide du paradigme fonctionnel,
et la comparaison de son efficacité avec celle obtenue avec une approche impérative.
 L\textquotesingle algorithme de détection sera évalué sur les critères de la rapidité 
d\textquotesingle exécution, ainsi que la précision de la détection.
 De fait, les approches impératives et fonctionnelles seront comparées sur ces deux aspects.

\end{abstract}

\section{Introduction}

La tonalité d\textquotesingle une œuvre musicale se caractérise par
l\textquotesingle ensemble des sons formant une même gamme diatonique. 
A la différence de la gamme, où les sons se succèdent de façon contigüe,
la tonalité (ou ton) regroupe des sons qui peuvent être disjoints et/ou superposés \citep{AD}.
En conséquence, nous nous intéressons à l\textquotesingle analyse de mélodies polyphoniques, 
où plusieurs notes peuvent être jouées en même temps.

L\textquotesingle algorithme utilisé pour la détection de la tonalité repose en partie
sur la solution proposée par Ibrahim Sha\textquotesingle ath lors 
de la conception du logiciel KeyFinder \citep{IS}.

TODO

La précision de la détection est évaluée à l\textquotesingle aide d\textquotesingle une base 
de données, constituée de 1000 fichiers musicaux au format wav, dont les tonalités sont connues
et inscrites dans un fichier csv. Ces fichiers sont identiques à ceux utilisés par 
Ibrahim Sha\textquotesingle ath dans le cadre de sa recherche.

TODO

\citep{SP} \citep{AT} \citep{JP}

\section{Considérations théoriques}

Le signal audio est premièrement extrait du fichier wav, puis la moyenne entre les 
deux canaux est effectuée si le fichier a été enregistré en stéréo. En effet il n\textquotesingle
est pas nécessaire de prendre en compte le panoramique puisque celui-ci n\textquotesingle a que 
peu d\textquotesingle influence sur la mélodie dans le domaine spectral. 
Etant donné que les notes jouées sont uniquement caractérisées par leur fréquence fondamentale,
il n\textquotesingle est pas nécessaire de considérer l\textquotesingle entièreté du spectre
du fichier musical. De fait, la fréquence d\textquotesingle échantillonnage est abaissée à un dixième 
de la fréquence standard (4410 Hz), mais ce sous-échantillonnage est susceptible de provoquer des
phénomènes d\textquotesingle aliasing. Contrairement à la solution de Sh\textquotesingle ath, 
qui gère les problèmes d\textquotesingle aliasing sonore par l\textquotesingle application 
d\textquotesingle un filtre passe-bas, une approche plus simpliste et plus rapide se limiterait à
l\textquotesingle application d\textquotesingle une moyenne mobile sur une fenêtre glissante
de taille arbitraire. L\textquotesingle avantage de la méthode est de bénéficier d\textquotesingle effets
passe-bas sans devoir effectuer de calculs dans le domaine fréquentiel. Pour ce qui est de la taille
de la fenêtre temporelle, il s\textquotesingle agit d\textquotesingle un paramètre fixé durant
l\textquotesingle évaluation de l\textquotesingle algorithme.

TODO - suite

\subsection{}

\section{Implémentation en Clojure}

TODO

\subsection{Implémentation en Python ?}

TODO - Ajouts de modèles de Markov cachés pour la gestion de l\textquotesingle aspect 
temporel ? Approche machine learning ?

\section{Résultats}

\subsection{Rapidité d\textquotesingle exécution}

TODO

\subsection{Précision des algorithmes}

TODO

\footnotesize
\bibliographystyle{apalike}
\bibliography{thesis}


\end{document}